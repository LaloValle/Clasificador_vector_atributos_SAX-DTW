
\hfill\break
\justifying
La inherente naturaleza descriptiva con una noción de ordenamiento, especificamente temporal, de las series de tiempo, les mantiene presente en casi cualquier tarea que involucra un proceso cognitivo humano[4], muchos otros fenómenos recurrentes en la naturaleza y cantidad de ámbitos de interes humano como negocios, economía, ingeniería, medio ambiente, medicina y otras áreas de investigación científica que recolectan datos en forma de secuencias dependientes del tiempo[5].

\hfill\break
\justifying
La tarea de clasificación de series temporales o \textit{Time Series Classification}(TSC), consiste en el entrenamiento de un clasificador en un conjunto de entradas de entrenamiento en unos casos específicos, en donde cada caso contiene un conjunto ordenado de atributos en valores reales y una etiqueta de clase[6]. Se trata de un tema ampliamente estudiado y de gran interés en un extenso rango de áreas como \textit{data mining}, estadística, procesamiento de señales, ciencias ambientales, bilogía computacional, procesamiento de imágenes, quimiometría, etc[6]. Representando tal importancia e influencia, que cualquier problema de clasificación que utiliza datos registrados tomando en consideración una noción de ordenamiento, puede ser transformado a un problema de TSC[4].

\hfill\break
\justifying
Con los años, investigadores han invertido un gran esfuerzo en estudios para el desarrollo de modelos que resuelvan el problema planteado por TSC y cada vez con una mayor precisión. Con la creciente disponibilidad de datos temporales[4] y archivos como \textit{UCR Time Series Archive}[7], ha desencadenado el crecimiento en el número de algoritmos propuestos para TSC, implementando técnicas que utilizan modelos de \textit{Machine Learning} y diferentes medidas de distancia, técnicas para la transformaciones del espacio de los datos[6], técnicas de \textit{ensembling}(entrenamiento de un conjunto de modelos menores que integrando sus salidas resultan en un mejor modelo)[8], hasta nuevas alternativas de modelos \textit{Deep Learning}, en la tendencia para muchas áreas de la IA, con un influencia relativamente nueva pero creciente en lo referente al problema de TSC[4].

\hfill\break
\justifying
Refiriendo a la problemática, la solución embebida como herramienta de apoyo a la comunicación verbal para personas con afecciones del habla, existen cantidad de artículos, \textit{papers} y proyectos que funcionan como estado de arte a la solución pero con un contraste importante en los objetivos entre este trabajo. La principal necesidad que atienden los trabajos[9,10,11,12] es la implementación de una solución tecnológica que se enfoca en traducir el lenguaje de señas del respectivo país, a discurso hablado, pero escencialmente las circunstancias en las que se plantean las problemáticas con este trabajo son distintas. La solución propuesta cuenta con un contexto muy distinto en el concepto de la traducción, implementando el código motriz como principal conjunto símbolico, particular y único a la solución, mostrando evidente la diferencia con la extendida población que comunica el lenguaje de señas.

\hfill\break
\justifying
Otro punto de constraste se encuentra en el ámbito de implementación y usuarios objetivos. Las tecnologías del estado de arte sirve para comunicar desde la comunidad con dominio en el lenguaje de señas, hacia la sociedad general en un contexto cotidiano. La solución que se propone en cambio, se muestra circunstancial pues funciona como herramienta de primer contacto con pacientes que han sufrido recientemente el accidente o lesión, y son incapaces de comunicarse pero en sus condiciones previas no se encontraban limitados en la expresión oral.

\hfill\break
\justifying
Enfocandose principalmente en el objetivo que atiende este artículo, el problema de TSC para gestos motrices, se distingue un \textit{paper} que atiende en profundidad y de manera particular, la identificación de gestos motrices utilizando un sensor acelerómetro para el muestreo de las fuerzas en los 3 ejes durante la ejecución del patrón con una mano[13]. En el trabajo Mezari y Maglogiannis, utiliza un \textit{smartwatch} Pebble como elemento sensor, exponiendo el objetivo del trabajo como una examinación del uso de dispositivos básicos como \textit{smartphones} y \textit{wearables}, para el reconocimiento confiable de gestos simples y naturales, proponiendo además una metodología para mejorar el desempeño y precisión.